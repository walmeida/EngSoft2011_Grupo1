\documentclass[12pt,letterpaper]{article}



% Page margin settings
\setlength{\textwidth}{6.5in} \setlength{\textheight}{9.0in}
\setlength{\topmargin}{-.25in} \setlength{\oddsidemargin}{0.0in}
\setlength{\evensidemargin}{0.0in}


% For encapsulated postscript figures
%\usepackage{epsfig}
% For figures in other image formats
\usepackage{graphicx,psfrag}
\usepackage{indentfirst}
\usepackage[brazil]{babel}
\usepackage[utf8]{inputenc}
\usepackage{hyperref}
%\usepackage[a4paper, tmargin=2.5cm, bmargin=2.5cm, lmargin=2.5cm, rmargin=2.5cm]{geometry}


\begin{document}

\sloppy

\title{
{\Large Instituto de Matemática e Estatística} \\
{\large Universidade de S\~ao Paulo, Brasil} \\
\vspace{2cm}
{\bf Academic Devoir}
}


\author{
André Satoshi\\
António Castro\\
Bruno Padilha\\
Gustavo Coelho\\
Luciana Kayo\\
Susanna Rezende\\
Suzana Santos\\
Vinicius Rezende\\
\vspace{2cm}
Wallace Almeida\\
{\small Professor: Marco Gerosa}\\
{\small Disciplina: MAC0332 Engenharia de Software}
\vspace{2cm}
{\small Versão: 2.0}
}

\date{\today}

\maketitle

\thispagestyle{empty}

%\begin{abstract}
%\end{abstract}

\pagebreak

\tableofcontents


%%%%%%%%%%%%%%%%%%%%%%%%%%%%%%%%%%%%%


%\pagebreak
%\section{Introdução}


\pagebreak
\section{Analista}

\subsection{Visão}
``Academic Devoir'' é um sistema a ser desenvolvido para uso em plataforma Web,  utilizando Java como linguagem básica de desenvolvimento, e adotando como tecnologias au\-xi\-liares os frameworks vRaptor e Hibernate.

O sofware tem como característica principal a gestão de tarefas acadêmicas das turmas de variadas disciplinas, baseadas em listas de exercícios, por sua vez constituídas de questões em diversas categorias, como dissertativa e ``Verdadeiro ou Falso'', e de correção automática.

O uso do sistema será destinado a professores e alunos. Os primeiros podem elaborar as atividades e visualizar a correção automática efetuada. Já os últimos poderão exibir as atividades disponíveis e solucioná-las.

%\pagebreak
%\subsection{Requerimentos do Sistema (System-Wide Requirements)}


\pagebreak
\subsection{Casos de Uso}
De acordo com o tipo de usuário (aluno ou professor), existirão tarefas que devem ser atendidas pelo sistema. Podemos listar algumas das possiveis utilizações:
\begin{itemize}
\item {Professor gerencia alunos}
\item {Professor cadastra disciplina}
\item {Professor cadastra turma}
\item {Professor cadastra lista de exercício}
\item {Professor cadastra questões}
\item {Professor lista respostas de questão}
\item {Professor pode reordenar questões de uma lista}
\item {Professor determina o tipo de matricula}
\item {Aluno se cadastra}
\item {Aluno se matricula em uma turma de uma disciplina}
\item {Aluno entra em uma disciplina e visualiza as listas}
\item {Aluno resolve uma lista de exercícios}
\end{itemize}


\pagebreak
\subsection{Modelo de Casos de Uso}


\pagebreak
\subsection{Glossário}

  {\bf G}\\
    Gerenciar: Entende-se que um usuário está apto a gerenciar uma dada
  estrutura do sistema quando possui permissões e interfaces para inclusão,
  edição, visualização e remoção de elementos dessa estrutura. 
    
  {\bf T}\\
    Turma: Conjunto composto por alunos (e monitor) que cursam uma disciplina 
    determinada oferecida por um professor em um dado semestre. Podem
    existir mais de uma turma, com professores distintos, para uma mesma 
    disciplina e semestre.  
    
  {\bf U}\\
    Usuário: Aquele que interage e utiliza o sistema. Nas especificações
    atuais, pode ser um professor ou um aluno.


\pagebreak
\section{Gerente}

\subsection{Plano da Iteração}

\vspace{1cm}
{\large {\bf Objetivos da iteração}}
\vspace{0.5cm}

Após as 3 primeiras semanas de ênfase na modelagem e na adaptação as práticas 
de desenvolvimento, inicia-se a segunda iteração do projeto junto a mudança de 
papéis dos integrantes.
Os objetivos propostos para a atual fase de desenvolvimento são:

\begin{itemize}
\item{} Complementar e aprimorar as funcionalidades elaboradas durante a primeira iteração;
\item{} Estabelecer os niveis de privilégio aos tipos de usuário;
\item{} Iniciar a integração entre os diversos módulos de uso do sistema;
\item{} Implementar o gerenciamento das turmas e disciplinas por parte dos professores;
\item{} Acrescentar mais informações a documentação do projeto.
\end{itemize}

Não será exigido até o momento que o sistema apresente funcionalidades 
completas com relação a elaboração de questões e de listas de exercícios por 
parte dos professores e a resolução destas por partes dos alunos.





\vspace{1cm}
{\large {\bf Atribuições de trabalho}}
\vspace{0.5cm}

A cada iteração, os cargos e responsabilidades devem ser trocados. As tarefas foram atribuídas conforme o papel dos integrantes. Dependendo da dificuldade da tarefa e de sua relevância para o desenvolvimento do projeto, todos os integrantes devem contribuir.

Segue uma lista dos participantes e de seus respectivos itens de trabalho, para essa segunda iteração: 




\begin{itemize}

\item {}André Satoshi Fujii de Siqueira (Analista de Qualidade)\\
\vspace{-0.5cm}
\begin{itemize}
\item{}Testes de Unidade para os CRUDS desenvolvidos.
%Divisão das tarefas de implementação
%Implementação do CRUD de questões e listas de exercício.
\end{itemize}
\vspace{0.5cm}

\item {}Antonio Junior (Desenvolvedor)\\
\vspace{-0.5cm}
\begin{itemize}
\item{}Divisão das tarefas de implementação
\end{itemize}
\vspace{0.5cm}

\item {}Bruno Padilha (Arquiteto)\\
\vspace{-0.5cm}
\begin{itemize}
\item{}Modelagem das classes do sistema.
\end{itemize}
\vspace{0.5cm}

%Implementação do cadastro de uma lista de exercícios.
\item {}Gustavo Coelho (Analista de Requisitos)\\
\vspace{-0.5cm}
\begin{itemize}
\item{}Levantamento de requisitos
\end{itemize}
\vspace{0.5cm}

%Implementação do login no sistema
\item {}Luciana Kayo (Desenvolvedor)\\
\vspace{-0.5cm}
\begin{itemize}
\item{}Divisão das tarefas de implementação\\
\item{}Formatação das páginas com css.
\end{itemize}
\vspace{0.5cm}

\item {}Susanna Rezende (Documentador)\\
\vspace{-0.5cm}
\begin{itemize}
\item{}Estudar documentação OpenUP e instruir os integrantes do grupo sobre como deve ser a documentação.\\
\item{}Acompanhar a documentação do projeto e dar as orientações necessárias.\\
%\item{}Ajudar a refatorar e comentar o código seguindo o padrão Javadoc.\\
\item{}Reunir a documetação de todas as partes do desenvolvimento.
\end{itemize}
\vspace{0.5cm}

\item {}Suzana de S. Santos (Desenvolvedor)\\
\vspace{-0.5cm}
\begin{itemize}
\item{}Divisão das tarefas de implementação
%Modelagem das classes do sistema
%Implementação do cadastro de uma resposta dada por um aluno no sistema.
\end{itemize}
\vspace{0.5cm}

\item {}Vinicius Rezende (Arquiteto)\\
\vspace{-0.5cm}
\begin{itemize}
\item{}Modelagem das classes do sistema.
%Implementação do CRUD de aluno, professor, disciplina e turma.
\end{itemize}
\vspace{0.5cm}

\item {}Wallace Faveron de Almeida (Gerente)\\
\vspace{-0.5cm}
\begin{itemize}
\item{}Monitoramento de algumas atividades realizadas pelos integrantes do grupo.\\
\item{}Coordenar a finalização das permissões nas classes e métodos.\\
\item{}Levantamento da visão do projeto e dos requisitos cumpridos até o momento.
\end{itemize}

\end{itemize}

\bigskip

Para maior detalhamento dos itens de trabalho, visite a página: 
\url{https://www.pivotaltracker.com/projects/355453#}

\vspace{1cm}
{\large {\bf Critérios de avaliação}}
\vspace{0.5cm}

Para avaliarmos o desenvolvimento, usaremos os testes de Unidade e a avaliação do cliente sobre o sistema.
Veja maiores detalhes sobre os testes na documentação do analista de qualidade.

\pagebreak
\subsection{Plano do Projeto}


\vspace{1cm}
{\large {\bf Organização}}
\vspace{0.5cm}

O trabalho no projeto é dividido nas seguintes áreas:


\begin{table}[ht!]
\begin{small} %tiny
    \begin{tabular}{| l | p{7cm} | p{5cm} |}
    \hline
    Identificação & Responsabilidades & Stakeholders\\
    \hline
    \hline
    Gerente do Projeto &
    Atribuições de caráter decisório e estratégico quanto aos rumos do projeto. &
    Wallace\\
    \hline
    Analistas &
    Definir e aprovar os requisitos e especificações de negócio do sistema. &
    Gustavo\\
    \hline
    Arquiteto do Projeto &
    Definir a arquitetura a ser utilizada no sistema. &
    Bruno e Vinícius \\
    \hline
    Documentação &
    Documentar &
    Susanna\\
    & & Colaboradores: Todos\\
    \hline
    Programadores &
    Implementar o sistema conforme as especificações. &
    Luciana, Suzana e Antònio\\
    & & Colaboradores: Todos\\
    \hline
    Testes &
    Padronizar os testes, homologar. &
    André\\
    \hline
    \end{tabular}
\end{small}
\end{table}


\vspace{1cm}
{\large {\bf Medições}}
\vspace{0.5cm}

A cada item de trabalho atribuímos pontos, que representam horas de dedicação (1 ponto equivale a uma hora).
Na página \url{https://www.pivotaltracker.com/projects/355453#}, temos os seguintes agrupamentos de itens de trabalho:
\begin{enumerate}
\item{Current (trabalho em desenvolvimento, concluído ou a ser desenvolvido em breve)}
\item{Backlog (trabalho a ser desenvolvido em breve)}
\item{Icebox (trabalho a ser desenvolvido sem data para início)}
\end{enumerate}

\vspace{1cm}
{\large {\bf Objetivos}}
\vspace{0.5cm}

Criar um sistema de resolução e correção de listas de exercício na Web.
O desenvolvimento consistirá de 3 iterações (cada uma com 3 semanas de duração) durante as quais serão implementadas as histórias:

\begin{itemize}
\item{Aluno se cadastra}
\item{Login de aluno/professor}
\item{Professor gerencia alunos (CRUD)}
\item{Professor cadastra disciplina}
\item{Professor cadastra turma}
\item{Professor cadastra lista de exercício}
\item{Professor cadastra questão de texto}
\item{Professor cadastra questão de V ou F}
\item{Professor cadastra questão de múltipla escolha}
\item{Professor cadastra questão de submissão de arquivo}
\item{Aluno se matricula em uma turma de uma disciplina}
\item{Professor lista respostas de questão de texto que ainda não foram corrigidas e pode corrigi-la}
\item{Professor pode reordenar questões de uma lista}
\item{Aluno entra em uma disciplina e visualiza as listas a serem feitas e as já corrigidas}
\item{Aluno resolve uma lista de exercícios}
\item{Professor determina o tipo de matricula em uma turma (imediato ou com prazo definido)}
\end{itemize}
Aguardamos mais informações do professor para completar a lista de histórias a serem implementadas.


\pagebreak
\subsection{Lista de Riscos}

Perguntar para o professor o que seriam os riscos do projeto.

\pagebreak
\subsection{Avaliação de status}

Aguardando feedback do cliente.

\pagebreak
\subsection{Lista de Itens de Trabalhos}

A lista dos itens de trabalho pode ser visualizada em: \url{https://www.pivotaltracker.com/projects/355453#}


\pagebreak
\section{Arquiteto}

\subsection{Caderno de Arquitetura}

Esboço do diagrama de classes\\
Obs.: Os nomes dados aos métodos e atributos nesse esboço são apenas descritivos. Na implementação, podemos ter nomes diferentes, seguindo os padrões do código.\\

%\begin{tabbing}

\begin{verbatim}
Esboço do diagrama de classes
Obs.: Os nomes dados aos métodos e atributos nesse esboço são apenas descritivos. Na implementação, podemos ter nomes diferentes, seguindo os padrões do código.

Classe abstrata
Usuário {
    ----- Atributos -----
    id
    nome
    login
    senha
    email
    ----- Métodos -----
    fazerCadastro(), 
    fazerLogin()  
    listarTurmas()
}

    Classes que implementam Usuário:

    Aluno {
        ----- Atributos -----
        lista de turmas (ids) [agregação n - n]
        ----- Métodos -----
        entregarLista()
        inscricaoNaTurma()
    }

    Professor {
        ----- Atributos -----
        lista de turmas (ids) [agregação n - n]
        ----- Métodos -----
        criarTurma()
        cadastrarDsiciplina()
        cadastrarNovaLsta()
        cadastrarQuestao()
    }

Turma {
    ----- Atributos -----
    id 
    disciplina 
    tipoDeMatricula 
    períodoDeMatrícula
    professor responsável
    lista de alunos (ids)
    lista de atividades (lista de lista de exercicíos) [agregação 1 - n]
    ----- Métodos -----
    listarAlunos() 
    listarListasDeAtividades() 
}

Disciplina {
    ----- Atributos -----
    id
    nome
    lista de turmas [composição 1 - n]
}

ListaDeExercicios {
    ----- Atributos -----
    id
    prazoDeEntrega
    visibilidade
    lista de questões (ids) [agregação n - n]
    lista de listas de respostas (lista de objetos do tipo ListaDeRepostas) [composição 1 - n]
    turma  
}

ListaDeRespostas {
    ----- Atributos -----
    aluno (id)
    estado
    lista de respostas. [composição  1 - n]
    ----- Métodos -----
    notaDaLista()
    [Sugestões de métodos:  envia(), salva()]
}


Classe abstrata
Resposta {
    ----- Atributos -----
    nota 
    corrigida?
    comentário
}

Classes que implementam Resposta:

    Multipla escolha {
        ----- Atributos -----
        lista das alternativas escolhidas pelo aluno.
    }
              
    Submissão de arquivo: {
        ---- Atributos -----
        arquivo enviado pelo aluno.
    }
              
     Texto {
        ----- Atributos -----
        String com a resposta dada pelo aluno.
     }


Classe abstrata
Questão {
    ----- Atributos -----
    id 
    valor
    enunciado 
    tags predefinidas
    tags definidas pelo usuário
}
  
Classes que inplementam Questões:
 
    Multipla escolha {
        ----- Atributos -----
        lista de alternativas
        alternativa correta (+ de uma?) { V/F }
     }
              
    Submissão de arquivo: {
	
    }
              
    Texto {
        ----- Atributos -----
        Resposta correta
     }


BancoDeQuestões {
    lista de questões
}


Classe sugeriada pelo André
QuestaoDaLista{
    Long id;
    Questao questao;
    Integer valor;
    Algum outro atributo que eu não pensei
}
\end{verbatim}
%\end{tabbing}

\pagebreak
\section{Desenvolvedor}

\subsection{Design}

\vspace{1cm}
{\large {\bf Design structure}}
\vspace{0.5cm}


%[Describe the design from the highest level. This is commonly done with a diagram that shows a layered architecture.]


Subsystems

%[Sub-system1]

%[Describe the design of a portion of the system (a package or component, for instance). The design should capture both static and dynamic perspectives.

%When capturing dynamic descriptions of behavior, look for reusable chunks of behavior that you can reference to simplify the design of the requirement realizations.

%You can break this section down into lower-level subsections to describe lower-level, encapsulated subsystems.]


\vspace{1cm}
{\large {\bf Patterns}}
\vspace{0.5cm}


%[Pattern1]

Overview

%[Provide an overview of the pattern in words in some consistent form. The overview of a pattern can include the intent, motivation, and applicability.]

Structure

%[Describe the pattern from a static perspective. Include all of the participants and how they relate to one another, and call out the relevant data and behavior.]

Behavior

%[Describe the pattern from a dynamic perspective. Walk the reader through how the participants collaborate to support various scenarios.]

Example
%[Often, you can convey the nature of the pattern better with an additional concrete example.]



\vspace{1cm}
{\large {\bf Requirement realizations}}
\vspace{0.5cm}


%[Realization1]

View of participants

%[Describe the participating design elements from a static perspective, giving details such as behavior, relationships, and attributes relevant to this realization.]

%Basic scenario

%[For the main flow, describe how instances of the design elements collaborate to realize the requirements. When using UML, this can be done with collaboration diagrams (sequence or communication).]

%Additional scenarios

%[For other scenarios that must be described to convey an appropriate amount of information about how the requirement behavior will be realized, describe how instances of the design elements collaborate to realize the requirement. When using UML, you can do this with collaboration diagrams (sequence or communication).]

%\pagebreak
%\subsection{Developer Test} %não entendi direito a difreneça ente isso e test case...

%\begin{itemize}
%\item{} Setup
%\item{} Inputs
%\item{} Script
%\item{} Expected Results
%\item{} Evaluation Criteria
%\item{} Clean-Up
%\end{itemize}




\pagebreak
\section{Testador}

A maioria dos testes realizados até o momento são testes nas classes Controller, ou seja, que testam se os Controllers chamam os métodos apropriados dos DAOs e se o usuário é redirecionado para as páginas certas. 

Há também testes que consistem em verificar se os diferentes objetos (Alunos, Professores, etc...) estão sendo corretamente inseridos em suas tabelas no banco de dados.

Para não afetar o banco de dados, todas os objetos que lidam com o banco de dados diratemente são mocks (imitações) dos objetos reais.

\subsection{Test Case}


\noindent Test Case ID - Test Case Name:\\
Descrição:\\ % [Describe the logical condition that the Test Case evaluates. Include the expected result.]\\
Pré-condições:\\ % [List conditions that must be true before this Test Case can start.]\\
Pós-condições:\\ % [List conditions that should be true when this Test Case ends.]\\
Dados requeridos:\\ %[Identify the type of data required for this Test Case.]\\


\pagebreak
\subsection{Test Log}

Produzido automaticamente. %Pode ser encontrado na pasta...
%TODO: Typically, the issue is not whether to produce the test log, but whether to keep a record, and where to keep the records.

%\pagebreak
%TODO: \subsection{Test Script} %não entendi esse item. Precisamos?




%\pagebreak
%\section{Bibliografia}\label{bibliografia}
%\bibliographystyle{plain}
%\bibliography{referencias}   % .bib


\end{document}
